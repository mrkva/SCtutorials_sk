\documentclass{beamer} 

\usetheme{Warsaw}
\usepackage{hyperref}
\usepackage[utf8]{inputenc}

\title{Supercollider pre začiatočníkov}
\author{Jonáš Gruska}
\institute[Royal Conservatory] % (optional)
{
		Institute of Sonology\\
		Den Haag, Netherlands
	}
\date{}

\begin{document}

\begin{frame}
\titlepage
\end{frame}

\section{Úvod}

\begin{frame}
\frametitle{Prečo programovať hudbu?}
	\begin{enumerate}
		\item Oslobodzovanie svojich postupov
		\item Viem ako čo funguje
		\item Originalita
		\item Väčšina hudobných jazykov je open-source
	\end{enumerate}
\end{frame}

\begin{frame}
\frametitle{Prečo programovať v Supercollideri?}
	\begin{itemize}
		\item Je úplne zadarmo a open-source
		\item Veľká a rastúca komunita používateľov
		\item Jednoduchá syntax a slušná dokumentácia
		\item Množstvo predprogramovaných hudobných objektov (UGens), v rôznych úrovniach zložitosti
		\item Vymyslený pre hudbu a hudobné použitie, menej zápasenia (C, Python, Java... )
		\item Pluginy -- Quarks 
	\end{itemize}
\end{frame}

\section{Supercollider}
% \subsection{Trochu histórie}
\begin{frame}
	\frametitle{Trochu histórie}
	\begin{itemize}
		\item Prvý release v roku 1996 - James McCartney
		\item Opensource od 2002
		\item GNU GPL licencia
	\end{itemize}
\end{frame}

\begin{frame}
	\frametitle{bla}
	\begin{itemize}
		\item
	\end{itemize}
\end{frame}

\begin{frame}
	\frametitle{bla}
	\begin{itemize}
		\item
	\end{itemize}
\end{frame}

\end{document}
